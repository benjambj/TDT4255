\documentclass{article}
\usepackage{graphicx}
\usepackage[colorlinks=true]{hyperref}
\usepackage[all]{hypcap}
\title{Exercise 1 - Report}
\author{Benjamin Bj\o rnseth}
\begin{document}
\maketitle 
\begin{abstract}
In this exercise, our task was to make a small combinatorial ALU. I made one which ended up supporting a small amount of basic arithmetic and logical operations, and which sets status flags according to the calculation's results. It includes a barrel shifter, which supports both arithmetic and logical shifts left and right. Functions that were not implemented include rotating and unsigned arithmetic.
\end{abstract}

\section{Introduction}
\label{sec:introduction}
Our task was to implement a combinatorial ALU in VHDL. It was to support some basic binary calculations, although it was not specified which were needed. This meant that we had to decide upon a subset of the different plausible functions we could implement, and also determine a mapping from the binary value in our function input signal to the calculation performed. Our ALU was also to include a barrel shifter as a component, as well as a register for storing our status. What values this register could assume after an instruction was performed and what it would mean was left for us to define. The ALU would then be integerated as a peripheral in an embedded system, which was to be uploaded on an FPGA \footnote{The FPGA was from the Spartan 6 family, contained on an Avnet S6LX16 development board}. 

We were also to perform three kinds of tests to make sure our design worked:
\begin{description}
\item[Functionality Simulation] Through the use of Modelsim, we were to assess our VHDL design's functionality. This was to be done by creating a VHDL test bench, and checking whether the output signals were what was expected.
\item[Timing Simulation] Using Modelsim, we were also to determine any timing constraints on our modelled system.
\item[FPGA Functionality Tests] We were also to test that our code, once uploaded onto the FPGA, did what it was supposed to do on the chip. We were to do this by communicating with a C driver program, which mapped our input values onto the ALU and sent back its output signals.
\end{description}

As I had not had any prior exposure neither to VHDL nor embedded systems programming, this task was not easy to come to grips with. I had, however, some experience from the first assignment given to us, and so I used this as a starting point to try figuring out how I was supposed to go about implementing this ALU. Apart from a few `must have' instructions, namely add and subtract, the final selection of supported functions was reached through an incremental approach. Basically, a function was implemented, tested, corrected, and then a new one was written. The carry flags were decided upon with guidance from the compendium; their semantics, however, were decided upon after rather careful consideration of the supported functions' semantics. 

\section{Solution}
\label{sec:solution}
The VHDL implementation of my solution is given in the file \emph{alu.vhd}. This file contains the description of my ALU entity, which has the following five signals:

\begin{enumerate} 
\item alu\_in\_a, which henceforth will be referred to as $A$.
\item alu\_in\_b, which henceforth will be referred to as $B$.
\item alu\_in\_funct, which henceforth will be referred to as $F$.
\item alu\_out, which henceforth will be referred to as $O$.
\item alu\_status, which henceforth will be referred to as $S$.
\end{enumerate}

Signals $A$, $B$ and $F$ are input signals, while $O$ and $S$ are output signals. $A$ and $B$ contain the function operands, $F$ is used for deciding which operation to perform, $O$ stores the result, and $S$ stores the status flag as described in \autoref{subsec:flags}.

\subsection{Supported Operations}
\label{subsec:ops}
The functions my ALU ended up supporting are listed in \autoref{tab:functions}. The gap between multiplication and nor results from an unimplemented division function, which was supposed to have function code $0x3$. The code values have been grouped together in logical groups for ease of remembrance, and not through some clever scheme to reduce circuit logic. Most functions are commutative; the only exceptions to this are the shift operators, in which $A$ contains the operand to shift and $B$ contains the shift amount.

\begin{table}[htbp]
  \centering
  \begin{tabular}{|l|l|}
    \hline
    {\bf Function}         & {\bf Function Code} \\ \hline
    addition               & 0x0 \\ \hline
    subtraction            & 0x1 \\ \hline
    multiplication         & 0x2 \\ \hline
    bitwise logical nor    & 0x4 \\ \hline
    bitwise logical xor    & 0x5 \\ \hline
    bitwise logical or     & 0x6 \\ \hline
    bitwise logical and    & 0x7 \\ \hline
    logical left shift     & 0x8 \\ \hline
    arithmetic left shift  & 0x9 \\ \hline
    logical right shift    & 0xA \\ \hline
    arithmetic right shift & 0xB \\ \hline

  \end{tabular}
  \caption{Supported functions. The \emph{Function Code} column describes the value of $F$ required to perform the calculation indicated in column \emph{Function}}
  \label{tab:functions}
\end{table}

\subsection{Status Flags}
\label{subsec:flags}
The ALU's different status flags are listed in \autoref{tab:flags}. They have been cleverly defined as different powers of two so that they can be or'ed together to perform conjunction of statuses. Thus, I shall use the phrase `the status flag is set to \emph{\textless{}flag name\textgreater{}}' to mean `$S \& \emph{\textless{}corresponding code\textgreater{}} \neq 0$'.
\begin{table}[htbp]
  \centering
  \begin{tabular}{|l|l|}
    \hline
    \textbf{Flag name} & \textbf{Code} \\ \hline
    Zero     & 0x1 \\ \hline
    Negative & 0x2 \\ \hline
    Overflow & 0x4 \\ \hline
    Carry    & 0x8 \\ \hline
  \end{tabular}
  \caption{ALU status flags}
  \label{tab:flags}
\end{table}
Their semantics are defined as follows:
\begin{itemize}
\item The `Zero' flag is set whenever $O = 0$.
\item The `Negative' flag is set when the operation just performed is of such a nature that a result should be interpreted as a signed number, and then when $O < 0$. In particular, it can be set after addition, subtraction, multiplication and arithmetic shifts, but not after the logic operations.
\item The `Overflow' flag is also used for signed calculations, and is set whenever the finiteness of the result register causes $O$ to unexpectedly change sign\footnote{The exception to this is multiplication, which cannot trigger overflow even though it is a signed arithmetic operation. The reason for this is that multiplication is already so severely limited in my implementation that reporting overflow would hardly matter. For more details, see \autoref{subsec:limitations}.}. Speficially, it may be set after addition, subtraction and arithmetic left shift\footnote{Arithmetic right shifts will never overflow, as they use sign extension; see \autoref{subsec:barrelshifter}}. For addition, overflow is set~$\iff~sign(A)=sign(B)\wedge{}sign(A)\neq{}sign(O)$. As subtraction is addition of a negative number, the previous equation holds with $sign(A)=sign(B)$ replaced with $sign(A)\neq{}sign(B)$. For arithmetic left shifts, the overflow flag is set if $O$ has a different sign from $A$. If the calculation does not change the sign, shifting bits out of $A$ is deemed as not being a problem with the signed interpretation, and as such overflow is not set.
\item The `Carry' flag is used for extending the result register on unsigned calculations. For my chosen set of instructions, this is only necessary for logical left/right shifts. In these cases, the carry bit is set to the last bit shifted out.
\end{itemize}

\subsection{The Barrel Shifter}
\label{subsec:barrelshifter}
The ALU does its barrel shifting by delegating its work to two barrel shifter components---one shifts left, the other one shifts right. The shifting is done whether the function we actually want to perform is a shift or not, and in the event that it actually is we map the corresponding component's output to our ALU output and calculate the status flags. 

Both components have the same structure, with four different signals:
\begin{itemize}
\item data\_in gives the operand to shift.
\item shift\_amount gives the amount by which to shift $D$.
\item data\_out\_unsigned stores the result of a logical shift.
\item data\_out\_signed stores the result of an arithmetic shift.
\end{itemize}
As you can see, the components perform both a logical and an arithmetic shift simultaneously. This choice of organization is arbitrary; the operations could just as well have been split into a signed and an unsigned shifter; one large shifter; or four small ones. 

The functions are performed by using the predefined VHDL functions $shl$ and $shr$ for the left- and right-shifter, respectively. These are overloaded on signed and unsigned operands, and depending on what their input is will perform either an arithmetic shift or a logical shift. Please see the enclosed source files \emph{barrel\_shift\_left.vhd} and \emph{barrel\_shift\_right.vhd} for more details. 

There is no status flag calculation logic contained in the shifter components; this is done in the ALU. This choice is deliberate, as it would feel artificial for an independent barrel shifter to include an ALU-specific output. The necessary calculations described in \autoref{subsec:flags} might just as easily be done in the ALU, anyways. Another reason for not performing them in the shifters is that the functionality is implemented through predefined functions, yielding no  possibility for doing these calculations more efficiently in the shifter anyways. If the shifter was on our critical path, swapping to a self-written shifter function might've been worth considering. This would yield more control over transitionary states, and we could've used this for generating the status flags---but it would also break the independency between the shifter and the ALU . As the tests in \autoref{subsubsec:timingsim} show, however, the shifter is not much of a concern timing-wise, and thus I saw no reason for tarnishing the relatively clean interface of the shifters with a status signal.

\section{Result}
\label{sec:result}
BLablabla
\subsection{Tests of the Implementation}
\label{subsec:tests}
The ALU I created was tested thrice. First, it went through a functionality simulation test, made to root out any logical errors in the VHDL wiring of the ALU. Then, I performed a timing simulation, in order to view the different combinatorial delays in my design and hopefully root out any potential problems. Finally, I performed tests of completed embedded system uploaded to the FPGA. 
\subsubsection{Functionality Simulation}
\label{subsubsec:funcsim}
The functionality simulation was done with the program Modelsim. A VHDL test bench\footnote{The testbench file is enclosed, and is named \emph{alu\_tb.vhd}} was created for driving the input signals of my ALU component to different values, and then checking in Modelsim whether $O$ and $S$ were assigned the expected values. In the test bench, $A$ and $B$ are set for 50 ns at a time, and this makes up a test case. Every test case is commented with the expected value of both $O$ and $S$. Checks for functional consistency were then manually performed by looking at the ALU output signals' values (see \autoref{fig:funcsim_shift_carry}), and then checking whether they matched the comments for the given test case or not. 
\begin{figure}[htbp]
  \centering
  \includegraphics[scale=0.5, bb=0 0 400 200]{figures/func_sim_wave_shift_carry_test.png}
  \caption{\label{fig:funcsim_shift_carry} Testing function 0b1000 = 0x8 = logical left shift}
\end{figure}

These simulations ended in success. I was able to satisfy the expected output of the specified test cases.

\subsubsection{Timing Simulation}
\label{subsubsec:timingsim}

\subsubsection{Functionality Tests on the FPGA}
\label{subsubsec:funcfpga}

\subsection{Design Limitations}
\label{subsec:limitations}
\paragraph{Multiplication}

\subsection{Reusability}
\label{subsec:reusability}
Barrel shifter without status flag; other than that, not very much for which the issue makes sense to discuss (I think?).

\section{Discussion}
\label{sec:discussion}
Discussion of what went well, and what could've been done better. Things I might have thought about, and things I could have implemented. Maybe include focus and time lost on understanding our implementation tools, which resulted in less consideration left for e.g. function value => function mapping, supported functions, status flags etc.

\section{Conclusion}
\label{sec:conclusion}
Round-up of sorts. Future work?




\end{document}
