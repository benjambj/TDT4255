\documentclass[11pt]{article}

\usepackage{graphicx,amsmath,amssymb,subfigure,url,xspace}
\newcommand{\eg}{e.g.,\xspace} \newcommand{\bigeg}{E.g.,\xspace}
\newcommand{\etal}{\textit{et~al.\xspace}}
\newcommand{\etc}{etc.\@\xspace} \newcommand{\ie}{i.e.,\xspace}
\newcommand{\bigie}{I.e.,\xspace}

\title{Exercise 3 - Implementation of a Pipelined Processor}
\author{Thomas Martinsen and Benjamin Bj\o rnseth}

\begin{document}
\maketitle

\begin{abstract}
  Introduce the report's content. What were we supposed to do. What
  did we achieve. Future work? Conclusions? Short.
\end{abstract}

\section{Introduction}
\label{sec:introduction}
Introduce the task in a bit more detail. Write about the challenges in
the task. For us:
 - We were to create a pipelined processor
 - This requires division into pipeline stages 
 - This introduces potential hazards that must be taken care of, both control hazards and data
hazards.  
- We were given suggestions with regards to

\section{Solution}
\label{sec:solution} General chitchat. Write this after writing the
subsections!  When writing subsection content, if there's something
you'd like to describe that's too general to fit in it, consider
writing it here.

\subsection{Instruction Set}
\label{subsec:instructionset} Tables! Explanation

\subsection{Instruction Format}
\label{subsec:instructionformat} Tables! Explanation

\subsection{The Processor Architecture}
\label{subsec:processor} Diagram

\subsection{The Control Unit}
\label{subsec:controlunit} New diagram?

\subsection{The Hazard Detection Unit}
\label{subsec:hazarddetection} New diagram?

\section{Results}
\label{sec:results} General chitchat To test the design, we tested
it. Write this after writing the subsections!

\subsection{Tests}
\label{subsec:tests}

Blablabla General information about test cases - what they do, what
they test and how they test it

\begin{itemize}
\item Test case 1 - Fibonacci
\item
\end{itemize}

\subsubsection{Functionality Simulation}
\label{subsubsec:funcsim} Images showing that these were
OK. Explanation for every image about relevant signals

\subsubsection{Timing Simulation}
\label{subsubsec:timingsim}

We had trouble with timing blablabla setup time not was not respected
with regards to status_register_out signal - not stable in time. This
was because we had the simulation clock period set to 10ns,
corresponding to 100MHz, which turned out to be too fast for our
processor. When we turned the period up to 15 ns, corresponding to the
clock frequency of the FPGA, the timing simulation results were
successfully completed. Blablabal... output from synthesis indicates
that our processor's maximum clock frequency is 78MHz-ish, which
corresponded nicely to our test results.

\subsubsection{Functionality Tests on the FPGA}
\label{subsec:funcfpga} Screenshots from console output, suggesting
the success of the tests when run on the FPGA.

\section{Discussion}
\label{sec:discussion}


\subsection{Design Limitations}
\label{subsec:limitations}

\subsection{Reusability}
\label{subsec:reusability}


\section{Conclusion}
\label{sec:conclusion}

\end{document}
