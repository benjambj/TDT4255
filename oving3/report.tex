\documentclass[11pt]{article}

\usepackage{graphicx,amsmath,amssymb,subfigure,url,xspace}
\usepackage[colorlinks=true]{hyperref}
\newcommand{\eg}{e.g.,\xspace} \newcommand{\bigeg}{E.g.,\xspace}
\newcommand{\etal}{\textit{et~al.\xspace}}
\newcommand{\etc}{etc.\@\xspace} \newcommand{\ie}{i.e.,\xspace}
\newcommand{\bigie}{I.e.,\xspace}

\title{Exercise 3 - Implementation of a Pipelined Processor}
\author{Thomas Martinsen and Benjamin Bj\o rnseth}

\begin{document}
\maketitle

\begin{abstract}
  In this exercise, we were to design and implement a pipelined
  processor. Our implementation consisted of three main pieces: The
  toplevel processor architecture, which performs calculations and
  operates on the register file and memory; A control unit, which is
  responsible for interpreting instructions and driving the processor
  accordingly; And a hazard detection unit, whose task is to detect
  and correct data hazards. The design's critical path ended up a bit
  long, but it was still able to function correctly when physically
  realized on an FPGA evaluation board.

  Introduce the report's content. What were we supposed to do. What
  did we achieve. Future work? Conclusions? Short.
\end{abstract}

\section{Introduction}
\label{sec:introduction}
The goal of this exercise was to design a pipelined processor, and
implement it using VHDL. Once completed, the implementation was to be
uploaded to an FPGA evaluation board\footnote{The evaluation board was
  an Avnet S6LX16.}. This board thus served as the targeted platform
on which the design had to function in the end. 
vv
When designing the processor, several choices had to be made. For
instance, we had to decide upon our processor's supported instruction
set, as well as the format of these instructions. Next, the
processor's architecture had to be specified. This included how the
data and control path should look, and how we should split the
processor's datapath into separate pipeline stages. Having made this
split, we had to check if any hazards could arise by having several
instructions in the pipeline simultaneously. These hazards would then
have to be handled, either by documenting them and disallowing code in
which they occur, or adding special hardware for coping with the
hazards.

\begin{table}[htbp]
  \centering
  \begin{tabular}{|c|c|c|c|c|c|c|c|}
    \hline
    {\bf name} & opcode & unused & funct & imm & Rd & Rb & Ra \\ \hline
    {\bf bits} & 31--29 & 28--24 & 23--20 & 19--12 & 11--8 & 7--4 & 3--0 \\ \hline
  \end{tabular}
  \caption{The suggested instruction format.}
  \label{tab:suggestedFormat}
\end{table}

We were provided with suggestions regarding nearly all of these design
issues. The suggested instruction format is shown in
\autoref{tab:suggestedFormat}. As for the instruction set, we
were required to provide the following instructions:
\begin{itemize}
\item Instruction(s) which use the ALU.
\item An instruction which loads an immediate value to a register.
\item An instruction which loads the value at a memory location to a register.
\item An instruction which stores the value in a register to a memory location.
\item A branch instruction.
\end{itemize}

We were also given suggestions with regards to the processor
architecture. When designing our processor, we decided to make use of
nearly all these suggestions. Specifically, our processor architecture
is so much alike the suggested one that the schematics are the same. 
- blablabla you can see it in sec:processorarch, 
(- only diff is control unit/hazard unit - mention this in solution? 
(- also diff in instr. format - mention in solution!)


\section{Solution}
\label{sec:solution} 
In our solution, the architecture is a four-stage pipelined
processor. The different stages are:
\begin{description}
\item[Fetch] In this stage, the instruction stored at the address
  contained in the program counter is fetched, and stored in an
  instruction register.
\item[Decode] In this stage, the instruction in the instruction
  register is decoded. The opcode is sent to a control unit to
  determine the values of control signals, the operands in the
  instruction is read from the register file, and the rest of the
  instruction is forwarded to thej next stage. 
\item[Execute] In this stage, any ALU operation required is performed,
  and memory loads or stores are issued.
\item[Write-back] In this stage, results which are to be stored in the
  register file are propagated to it. The value to be stored is not
  actually updated until the next cycle, however.
\end{description}
In this design, there is no potential for structural hazards: no two
instructions use the same hardware components at the same time. There
are, however, possibilities for both data hazards and control hazards. 

The data hazards occur when one instruction earlier in the pipeline
uses the value of a register which is written by an instruction later
in the pipeline. There are three cases where this can happen:
\footnote{In our register file, register zero has the constant value
  zero. Thus, the hazards described here does not hold for register
  zero.}
\begin{enumerate}
\item The instruction in the decode stage loads operands from a
  register which is written to by the instruction in the write-back
  stage. 
\item The instruction in the execute stage uses operands from a
  register which is written to by the instruction in the write-back
  stage.
\item The instruction in the decode stage is a branch which reads the
  status register, and the instruction in the execute stage is an ALU
  instruction which writes to the status register.
\end{enumerate}
To deal with these issues, we implemented logic which detects the
hazard and forwards the correct value to the instruction using
it. This logic is described in \autoref{subsec:hazarddetection}.

In our design, a single control hazard can occur: If a branch is
taken, the instruction being loaded into the instruction register in
the fetch state will be the one following the branch instruction and
thus most likely not be the one we actually want to execute next. As
such, we need a way to mark the instruction as invalid. We solved this
by maintaining a validation bit.

As we designed our processor based on our instruction set and format,
and not vice versa, we start by introducing these. Afterwards, we
describe the architecture of our implementation in three parts -
processor architecture, the control unit and the hazard unit.

\subsection{Instruction Set}
\label{subsec:instructionset} 
The instruction set of our processor is given in pseudo-assembly in \autoref{tab:instructionSet}. 

\begin{table}[htbp]
  \centering
  \begin{tabular}{|c|p{165pt}|}
    \hline
    {\bf Pseudo-assembly} & {\bf Semantics} \\ \hline
    ADD/SUB/MOV/AND r{\em D}, r{\em A}, r{\em B} &  Performs the given operation on the values in register number {\em A} and {\em B}, and stores the result in register number {\em D}. For the non-commutative operation SUB, it is important to note that {\em B} gets subtracted from {\em A}, {\ie} SUB r{\em D}, r{\em A}, r{\em B} corresponds to $D = A - B$. For MOV, only r{\em A} is used. \\ \hline
    LDI r{\em D} {\em value} & Stores {\em value} in register number {\em D}. \\ \hline
    LD r{\em D}, {\em address} & Loads the value located at memory address $address$ into register number {\em D} \\ \hline
    ST r{\em B}, {\em offset} & Stores the value in register number {\em B} at memory address $address$. \\ \hline
    BNZ {\em address} & If the ALU instruction last executed did not set the status register of the ALU to zero, instruction fetching will continue from instruction memory address {\em address}, {\ie} if the result of the previous computation was non-zero, we jump to address $address$. Otherwise, instruction fetching will resume as if no instruction had been executed. \\ \hline
  \end{tabular}
  \caption{The supported instruction set.}
  \label{tab:instructionSet}
\end{table}


\subsection{Instruction Format}
\label{subsec:instructionformat} 
The format of our instructions is tabulated in
\autoref{tab:instructionFormat}. To make both instruction decoding as
well as the mapping between the semantics of an instruction and its
binary format simple, it has a fixed size with fields designated to a
specific purpose across all instructions. 
\begin{table}[htbp]
  \centering
  \begin{tabular}{|c|c|c|c|c|c|c|}
    \hline
    {\bf name} & opcode & func & immediate & rD & rB & rA \\ \hline
    {\bf bits} & 31--29 & 28--25 & 24--12 & 11--8 & 7--4 & 3--0 \\ \hline
  \end{tabular}
  \caption{The instruction format of our processor.}
  \label{tab:instructionFormat}
\end{table}

The opcode field determines what kind of instruction is
executed. \autoref{tab:opcodeInstructionMapping} illustrates the exact
mapping. As we have five different instructions, three bits were
needed to distinguish these if we were to keep with our intention of
having no elaborate decoding scheme.

\begin{table}[htbp]
  \centering
  \begin{tabular}{|c|c|}
    \hline
    {\bf Opocode} & {\bf Instruction Performed} \\ \hline
    000 & ALU-operation \\ \hline
    001 & Load immediate \\ \hline
    010 & Load from memory \\ \hline
    011 & Store to memory \\ \hline
    100 & Branch not zero \\ \hline
  \end{tabular}
  \caption{The mapping between the value in the opcode field and the instruction performed.}
  \label{tab:opcodeInstructionMapping}
\end{table}

The function field has an effect only during the execution of an ALU
instruction, as this is the only instruction during which the status
register can be written and the ALU result is used further down the
pipeline. For an ALU-instructino, the mapping between the binary value
of the function field and the operation performed is given in
\autoref{tab:funcOperationMapping}. In any other case, the bits have
don't-care values.

\begin{table}[htbp]
  \centering
  \begin{tabular}{|c|c|}
    \hline
    {\bf Func} & {\bf ALU operation} \\ \hline
    0000 & MOV \\ \hline
    0001 & AND \\ \hline
    0110 & SUB \\ \hline
    0111 & ADD \\ \hline
    Otherwise & The ALU output is set to zero \\ \hline
  \end{tabular}
  \caption{The mapping between the value in the opcode field and the alu operation performed. The function field can take any value when executing a non-ALU instruction, as the result computed by the ALU has no effect.}
  \label{tab:funcOperationMapping}
\end{table}

The immediate field is used for several things: Holding the branch
address during Branch not zero; Holding the memory address during
loads and stores; and holding the immediate value during Load
immediate. When it is used as an address, as is the case for both
branches and memory accesses, the eight least significant bits are
used. When it is used as the immediate value to load into a register,
it is zero-extended to 32-bits.

The last three fields contain binary digits identifying which
registers to use during an instruction. When an instruction does not
use three registers, the ones it doesn't use assume don't-care values. 

\subsection{The Processor Architecture}
\label{subsec:processor} 

\begin{figure}[ht]
  \centering
  \includegraphics[scale=0.25]{figures/pipelined_architecture.png}
  \caption{\label{fig:processorArchitecture} }
\end{figure}

A schematic of our architecture is depicted in
\autoref{fig:processorArchitecture}. To the left of the instruction
memory is our program counter register, whose input comes from a
multiplexor choosing between either an incremented version of the
program counter, or the eight least significant bits of the
immediate. The multiplexor is controlled by the control unit, as
further explained in \autoref{subsec:controlunit}. The pipeline stages
are divided by the three large registers in the schematic. The output
from the instruction memory is stored in the first of these, segmented
as per the instruction format described in
\autoref{subsec:instructionformat}\footnote{The topmost part is the
  most significant bit, {\ie} the first segment is the opcode, the
  second is the function field {\etc}}. The other large registers are
the ones which seperate the decode and execute stage and the execute
stage and write-back stage. Their contents follow from what they are
connected to in the preceding stage. 

The tiny register above the instruction register holds a validation
bit, which is used to determine whether the instruction in the
instruction register is valid or not. It is used to avoid executing
the instruction following a branch when the branch is taken, and any
garbage value contained in the register when the processor is started.

The two multiplexors after the register file controls which data is
stored as potential input to the alu. These handle the data hazard
that occurs if the instruction in the write-back stage writes to a
register which the instruction in the decode stage uses as an
operand. This selection is described in more detail in
\autoref{subsec:hazarddetection}. 

The control unit governs which signals and components are active
during the execution of a given instruction. The hazard detection unit
controls the multiplexors which forward data in the case of data
hazards. These components are described in
\autoref{subsec:controlunit} and \autoref{subsec:hazarddetection},
respectively.

The multiplexor in front of ALU input A and B handles the data hazard
in which the instruction in the execute stage depends on the result of
the preceding instruction, now in the write-back stage. The output
from the multiplexor in front of ALU input A goes both as an input to
the alu as well as data memory, as this is the register which contains
the value used during a store.

The immediate field used to address memory during the execute stage is
the eight least significant bits of the immediate value. We thus
extract these bits before sending them to the data memory module.

The tiny register above the ALU is the status register, which stores a
bit which indicates whether the last result from an ALU calculation
was zero or not. The control signal entering it determines whether the
instruction currently in the execute-phase can write to it or not. The
multiplexor straight above it handles the potential data hazards
occuring when a branch instruction directly follows an ALU
instruction, by forwarding the current ALU status to the control unit
instead of the old value.

The right-most multiplexor selects the input to the register file,
choosing between the ALU output, the data read from memory, or the
immediate value. As the immediate value is only 13 bits, it is
zero-extended to 32 bits to fit the register file size. The selection
is controlled by the control bits in the pipeline register, which has
been passed along from the previous stage.

\subsection{The Control Unit}
\label{subsec:controlunit} 
The structure of our control unit is depicted in
\autoref{fig:controlunit}.

\begin{figure}[ht]
  \centering
  \includegraphics[scale=0.35]{figures/control_unit.png}
  \caption{\label{fig:controlunit} A schematic of our control unit, as a black box. The input signals are given on the left, and the output signals on the right.}
\end{figure}

It is a completely combinatorial circuit, and the outputs are
calculated as functions of the inputs. The control signal semantics
and their calculation formulae are given in \autoref{tab:controlunit}.

\begin{table}[htbp]
  \centering
  \begin{tabular}{|c|p{150pt}|p{150pt}|}
    \hline
    {\bf Output signal} & {\bf Semantics} & {\bf Calculation} \\ \hline
    {\bf Status reg WE} & Determines whether this instruction can write to the status register. & Opcode = ALU \\ \hline
    {\bf Dmem WE} & Determines whether this instruction can write to data memory & Opcode = STORE \\ \hline
    {\bf Regfile mux} & Selects the data to propagate to the register file & Opcode = ALU : Select ALU output, 
                                                                             Opcode = LDI : Select immediate, 
                                                                             Opcode = LOAD : Select memory output\\ \hline
    {\bf Regfile WE} & Determines whether this instruction can write to the register file & Opcode = ALU or LDI or LOAD \\ \hline
    {\bf PC mux} & Selects which program counter value to fetch instructions from. & Opcode = BNZ and status\_register = 0 : Select immediate , 
                                                                                                                    Else  : Select incremented value\\ \hline
    {\bf Next instr. valid} & Indicates whether the next instruction is valid or not. & Opcode = BNZ and status\_register = 0 and current\_instruction\_valid=1 : 0 ,
                                                                                                                                                        Else : 1\\ \hline
  \end{tabular}
  \caption{The semantics of the different control signals output from the control unit, and the formula used to calculate them.}
  \label{tab:controlunit}
\end{table}


\subsection{The Hazard Detection Unit}
\label{subsec:hazarddetection} 
The structure of our hazard detection unit is shown in
\autoref{fig:hazarddetection}.

\begin{figure}[ht]
  \centering
  \includegraphics[scale=0.35]{figures/hazard_unit.png}
  \caption{\label{fig:hazarddetection} }
\end{figure}

As the control unit, this is also a combinatorial circuit. Its output
signals' semantics and calculation are tabulated in
\autoref{tab:hazarddetection}.\footnote{The data hazards discussed
  here were introduced in \autoref{sec:solution}}

\begin{table}[htbp]
  \centering
  \begin{tabular}{|p{60pt}|p{150pt}|p{150pt}|}
    \hline
    {\bf Output signal} & {\bf Semantics} & {\bf Calculation} \\ \hline
    {\bf Forward status} & High when we have a data hazard on the status register & Opcode = BNZ and status\_register\_WE = 1 \\ \hline
    {\bf Forward regD to decode stage regA} & High when the same register A is loaded in the decode stage as is written in the write-back stage & write-back regD = decode regA and write-back regD /= reg zero and regfile\_WE = 1 \\ \hline
    {\bf Forward regD to decode stage regB} & \multicolumn{2}{|p{300pt}|}{The same as above except that register A is replaced with register B} \\ \hline
    {\bf Forward regD to exec stage regA} & High when the same register is alu operand A in the execute stage as is written in the write-back stage & write-back regD = exec operandA and write-back regD /= reg zero and regfile\_WE = 1 \\ \hline
    {\bf Forward regD to exec stage regB} & \multicolumn{2}{|p{300pt}|}{The same as above except that operand A is replaced with operand B} \\ \hline    
  \end{tabular}
  \caption{The semantics of different hazard detection unit output signals, and the formula used to calculate them.}
  \label{tab:hazarddetection}
\end{table}

\section{Results}
\label{sec:results} 

We will now present the results obtained by testing the functionality
of our piplined processor.  It was executed four different test
programs on the FPGA intended to validate the functionality of the
processor and the handling of the various hazards that may arise.  We
will start with reviewing the the pseudo code for the various test
programs that we have used.  The different hazards that may occur will
be specifically examined.


\subsection{Tests}
\label{subsec:tests}

The tests are presented in their own sections where the code is
examined in detail. This aims to provide a better understanding of
test programs before we look at the simulation results.

\subsubsection*{Fibonacci}
Our first program is a traditional algorithm to compute the Nth
Fibonacci number. This test is chosen to validate the general
functionality of the processor.
\begin{enumerate}
\setcounter{enumi}{-1}
\item LDI R1,1
\item LDI R3,1
\item LD R4,0 //load N from main memory
\item MOV R5,R3 //forward R3 from instruction 1
\item ADD R3,R2,R3
\item MOV R2,R5
\item SUB R4,R1,R4
\item BNZ 2 //we need to forward the status from the ALU
\item ST R2,0 //instruction will be marked as invalid
\item MOV R1,R1
\item BNZ 9 //we need to forward the status from the ALU
\end{enumerate}

\subsubsection*{Loads and stores}
We have chosen to create a special test to validate that the data
which is loaded from main memory are forwarded correctly.
\begin{enumerate}
\setcounter{enumi}{-1}
\item LD R1, 0
\item ST R1, 2 //forward R1 from instruction 0
\item LD R2, 2
\item ST R2, 3 //forward R2 from instruction 2
\item LDI R3, 1
\item MOV R3, R3 //forward R3 from instruction 4
\item BNZ 5 //we need to forward the status from the ALU
\end{enumerate}

\subsubsection*{pow2}
The third test case calculates numbers that is a power of two. The
purpose of this test is to validate that the forwarding works for all
the pipeline stages when we only use one registry during the
calculations.
\begin{enumerate}
\setcounter{enumi}{-1}
\item LDI R1, 1
\item ADD R1, R1, R1 //forward R1 from instruction 0
\item ADD R1, R1, R1 //forward R1 from instruction 1
\item ADD R1, R1, R1 //forward R1 from instruction 2
\item ADD R1, R1, R1 //forward R1 from instruction 3
\item ADD R1, R1, R1 //forward R1 from instruction 4
\item ADD R1, R1, R1 //forward R1 from instruction 5
\item ST R1, 2 //forward R1 from instruction 6
\item ADD R1, R1, R1 //forward R1 from instruction 6
\item ST R1, 0 //forward R1 from instruction 8
\end{enumerate}

\subsubsection*{if-else}
We have made an if-else test to demonstrate that the instructions that
are marked as invalid is actually not carried out. We have also added
an LDI instruction between SUB and the BNZ to show that the status
register works without any forwarding.
\begin{enumerate}
\setcounter{enumi}{-1}
\item LDI R1, 2 
\item LDI R2, 1
\item SUB R3, R2, R1 //forward from instruction 0 and 1
\item LDI R5, 5
\item BNZ 6 //we use the status register
\item ST R2, 0 //the store instruction is invalid
\item ST R1, 1
\item MOV R1, R1
\item BNZ 7
\end{enumerate}

\subsubsection{Functionality Simulation}
\label{subsubsec:funcsim}
The simulation of the test bench was done with ISim. This is a similar
simulation program like ModelSim only that it is newer and easier to
use. We will now go through the various test programs and see that the
signals are properly set.

\subsubsection*{Fibonacci}
\begin{figure}[!h]
\includegraphics[scale=0.5]{fibonacci_load_immediate_1_to_reg_1.png}
\caption{We are loading 1 immediate into R1.}
\label{fig:register_write}
\end{figure}

\begin{figure}[!h]
\includegraphics[scale=0.5]{fibonacci_load_parameter_from_memory.png}
\caption{We load 11 from memory address 0, into register R4. We have now selected which Fibonacci number to calculate.}
\label{fig:register_write}
\end{figure}

\pagebreak

\begin{figure}[!h]
\includegraphics[scale=0.5]{fibonacci_decrement_counter_status_reg_mux_out_not_zero_no_forwarding.png}
\caption{BNZ follows directly after an ALU instruction. This results in that the status from the ALU will be forwarded.}
\label{fig:register_write}
\end{figure}

\begin{figure}[!h]
\includegraphics[scale=0.5]{fibonacci_branch_taken_invalid_instruction_negated_control_signals.png}
\caption{When we subtracts R1 from R4, the result is 10. BNZ is carried out and the next instruction that is fetched is marked as invalid.}
\label{fig:register_write}
\end{figure}

\pagebreak

\begin{figure}[!h]
\includegraphics[scale=0.5]{fibonacci_counter_zero_branch_not_taken_forwarding_status_register.png}
\caption{When we subtracts R1 from R4, the result is 0. BNZ does not occur and the next instruction is valid.}
\label{fig:register_write}
\end{figure}

\pagebreak

\begin{figure}[!h]
\includegraphics[scale=0.5]{fibonacci_write_data_memory_write_enabled.png}
\caption{Writing to the main memory is enabled, and the calculated Fibonacci number is stored.}
\label{fig:register_write}
\end{figure}

\pagebreak


\begin{figure}[!h]
\includegraphics[scale=0.5]{fibonacci_reading_result_toplevel.png}
\caption{We read 89 from memory address 0. This is the eleven Fibonacci number which shows that the calculations are correct.}
\label{fig:register_write}
\end{figure}

\subsubsection*{Loads and stores}
\begin{figure}[!h]
\includegraphics[scale=0.5]{loadsandstores_load_forwarding_to_store.png}
\caption{When we load the data from main memory, it will be forwarded to the following storage instruction. One can notice that we found out that it is possible to change the output to both binary, decimal and hex.}
\label{fig:register_write}
\end{figure}

\begin{figure}[!h]
\includegraphics[scale=0.5]{loadsandstores_results_number_in_every_address_location.png}
\caption{Forwarding has functioned correctly, and the expected results are read from the toplevel.}
\label{fig:register_write}
\end{figure}

\subsubsection*{pow2}
\begin{figure}[!h]
\includegraphics[scale=0.5]{usereg1_forwarding_results.png}
\caption{We see that the forwarding signals are continuously on, since we only use R1 during the calculation.}
\label{fig:register_write}
\end{figure}

\subsubsection*{if-else}
\begin{figure}[!h]
\includegraphics[scale=0.5]{testbranch_no_forward_status_register_marks_following_invalid.png}
\caption{LDI follows an ALU instruction, so BNZ use the status register without forwarding the result. The branche is taken so the following storage instruction is marked as invalid.}
\label{fig:register_write}
\end{figure}

\begin{figure}[!h]
\includegraphics[scale=0.5]{testbranch_results_stored_in_2_not_0.png}
\caption{When we read from the main memory, we see that the invalid storage instructions have not been performed.}
\label{fig:register_write}
\end{figure}

\subsubsection{Timing Simulation}
\label{subsubsec:timingsim}

Before we carried out functionality tests on the FPGA, we tested that
the functions behaved correctly with respect to the Spartan
architecture.  This was achieved by performing timing simulations in
ISim.  During the simulations we experienced one problem that the
status register was not updated quickly enough.  As a result we used
time to consider optimizations we could do to improve how quickly the
control unit could be notified if a branch should be taken or not.
What we discovered during the simulation was that we had set the clock
frequency at 100MHz.  Our processor design synthesized only to support
a maximum of 78MHz.  So when we turned down the clock frequency to
66MHz, corresponding to the clock frequency of the FPGA, the processor
design functioned correctly.

\subsubsection{Functionality Tests on the FPGA}
\label{subsec:funcfpga} 
To verify the functionality of our pipelined processor after it had
been configured on the FPGA, we used the python script we had been
provided to initialize the instruction memory with our simulation
programs. We tested the four simulation programs on the FPGA.  Before
each test we initialized the instruction and data memory with
zero. This was done to ensure that old data could not ruin the
execution of the different programs.

\subsubsection*{Fibonacci}
\begin{figure}[!h]
\includegraphics[scale=1.0]{basic_design.png}
\caption{The processor was able to calculate the eleventh Fibonacci number when we performed the test on the FPGA.}
\label{fig:register_write}
\end{figure}

\subsubsection*{Loads and stores}
\begin{figure}[!h]
\includegraphics[scale=0.5]{loads_and_stores_in_sequence.png}
\caption{We see that forwarding of the data read from main memory works correctly, the following storage instruction have the expected behavior.}
\label{fig:register_write}
\end{figure}

\subsubsection*{pow2}
\begin{figure}[!h]
\includegraphics[scale=0.5]{fill_pipeline_with_usages_of_reg1_to_test_hazards.png}
\caption{Data and instruction memory is initialized with zero. We read 128 from the data memory after execution of the program which is the expected result.}
\label{fig:register_write}
\end{figure}

\subsubsection*{if-else}
\begin{figure}[!h]
\includegraphics[scale=0.5]{test_branch_works_when_no_hazard.png}
\caption{Nothing has been stored in memory address 0, but we load 2 from memory address 1.}
\label{fig:register_write}
\end{figure}

\section{Discussion}
\label{sec:discussion}
Our processor has a relatively low percentage of wasted cycles during
execution. There are few hazards our design cannot handle, and as such
an average of one instruction completed per cycle is achievable with
our supported instruction set. There is one case in which we lose an
instruction slot by inserting a no-op, however, and that is upon
encountering a branch which is taken\footnote{This is explained in
  \autoref{sec:solution}}. Thus we have a case of suboptimal
performance, and this would be a point deserving of further
investigation if the design was to be reconsidered. As we had to
consider the entirety of the architecture and design, however, we
decided to stick with the simplest solution to simplify the validation
of the system as a whole.

One alternative solution would have been to try and forward the jump
address to the instruction memory, so that the next instruction to
load would have been the correct one. This does, however, seem
optimistic, considering how branch decision making is already on our
critical path. Adding any more logic here would most likely extend
this, and we would have to increase our cycle period. Thus, even if
this was a plausible solution we might lose out on processor
performance due to the decrease in clock frequency. This would have to
be investigated further before a conclusion can be reached.

The way of resolving branches previously described would also be
slightly messy, as it might not be an intuitive way of resolving the
issue. A slightly more sophisticated approach would be to implement
branch prediction. This would not completely remove the need of
inserting no-ops in the pipeline, but it might reduce it. For
instance, in a loop which jumps backwards we'd get a no-op for every
iteration, whereas if branch prediction was implemented we'd only get
one or two. As in the previous case, one would have to check whether
this increased the minimal clock cycle time, and weigh the
alternatives after having made precise measurements. 

\newline{}

Another point at which we had to choose between simplicity and
optimality was the design of our instruction format. In our current
design, we have a simple format whose structure and size is constant
across all instructions. This leads to very simple decoding: All
fields are treated as a value in its own right, and there's no need to
decode based on the instruction's content. The downside of this is
that we end up with a lot of wasted space in our instruction
memory. None of our instructions use all the fields, for instance, and
some of the fields can be collapsed into smaller ones. If we allowed
the instruction format to vary in size and allowed for increased
decoding complexity, we could achieve denser code. As an example, the
ALU operations does not use the immediate field. If the instruction
simply didn't contain this field, we would save 13 bits per ALU
instruction in the program. Since our instruction memory space is
limited, this could be a worthy tradeoff. As was mentioned, however,
the complexity that comes with this solution is significant. We would
have to support fetching of non-alligned memory words, and decode the
instruction based on its content. Not only would the design get
messier, it might also make the processor slower. One can thus not
reach any conclusion in the general case with regards to the
optimality of one option versus the other.

\subsection{Design Limitations}
\label{subsec:limitations}
Our processor has a very limited set of use cases, as our instruction
set is small. What is provided might be enough to meet certain needs,
but others might find it to be lacking. As such one should analyze the
scenario in which it is to be used, to discover minimum
functionality. Certain general improvements might be implemented,
however, and are described below.

One significant limitation is our load and store instructions, which
address memory based on the immediate field. There is as such no room
for dynamic addressing, and arrays cannot be supported. One way to fix
this particular issue could be to change these instructions to instead
use the eight least significant bits of the contents of register B as
the address. This would allow for a program which calculates the
addresses to use. This should be implementable with little
effort. Extending the ALU with more operations would also be
plausible, as we have room in our instruction for addressing sixteen
but we only use four. If we only added ones with an execution time
lower than the cycle time, this could also be effortlessly completed.

\newpage{} 

Another issue is the speed of the processor. It runs at slightly more
than one third of the maximum clock frequency of the FPGA, and for our
purposes this might be enough. That's not to say it cannot be
improved, though. Further analysis of our critical path might reveal
actions which could be taken to increase the maximum clock frequency
without hampering the functionality of the processor, and if profiling
deemed this to be necessary for applications to run at an acceptable
pace one could consider improving the performance.

\subsection{Reusability}
\label{subsec:reusability}
Our processor has been implemented with the use of named constants and
generics in the different components, instead of magical numbers. This
makes deploying it to environments with different memory and bus sizes
easier. For instance, if we wanted adapt the processor for use on a
64-bit architecture, adjusting these values and making a few
modifications to the ALU should be all that is needed.

We have also seperated the hazard detection unit and control unit into
individual modules. This might make it easier to reuse these
components in different designs, where there is a lot of overlap
between functionality and hazards.

\section{Conclusion}
\label{sec:conclusion}
We have in this exercise implemented a three stage pipelined
processor. The main challenge we faced was to test that we had handled
all the hazards that can occur in our processor design. This led to
the development of several test programs so we could do a detailed
analysis of the behavior of the processor.  Our experiences from the
implementation of a multi-cycle processor made this exercise
easier. This is because we have now got more experience with how the
clock affects the digital design. A good lesson we learned from the
implementation of the pipelineded processor was how overclocking could
make the circuit unstable.  These experiences were obtained from the
timing simulations we performed, it nevertheless gave us a better
insight in the properties one must consider when implementing a
processor. How much work is done in a cycle against how long a cycle
takes. The execution of the test programs on the FPGA was a success.

\end{document}
